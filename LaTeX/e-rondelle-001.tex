\documentclass[francais,a4paper,oneside]{article}
\usepackage[T1]{fontenc}
\usepackage[utf8]{inputenc}
\usepackage[francais]{babel} 
\usepackage{graphicx}
\usepackage{lscape}
\usepackage{url}
\usepackage{amssymb}
\usepackage{multicol}
\usepackage{amsmath}
\usepackage{listings}
\usepackage[svgnames,dvipsnames]{xcolor}
\usepackage{lastpage}
\usepackage{babel}
\usepackage{eurosym}
\usepackage{lettrine}
\usepackage{fancyhdr}
\usepackage{setspace}
\usepackage{lmodern}
%\usepackage{sectsty}
%\usepackage[12pt]{extsizes}
\usepackage{listings}
%\usepackage[squaren,Gray]{SIunits}

%\areaset{250mm}{155mm}
%%%% debut macro %%%%
\newenvironment{changemargin}[2]{\begin{list}{}{%
\setlength{\topsep}{0pt}%
\setlength{\leftmargin}{0pt}%
\setlength{\rightmargin}{0pt}%
\setlength{\listparindent}{\parindent}%
\setlength{\itemindent}{\parindent}%
\setlength{\parsep}{0pt plus 1pt}%
\addtolength{\leftmargin}{#1}%
\addtolength{\rightmargin}{#2}%
}\item }{\end{list}}
%%%% fin macro %%%%


\usepackage[a4paper]{geometry}
\geometry{hscale=0.75,vscale=0.8,}
\geometry{margin={8mm,8mm}}
\geometry{vmargin={3mm,16mm}}
\makeatletter
\newcommand{\noun}[1]{\textsc{#1}}
\setlength{\parindent}{20pt}
\setlength{\parskip}{1ex plus 0.5ex minus 0.2ex}

\newcommand\largE{\@setfontsize\large{16}{20}} 


%---SORTIES--------------------------------------------------------------------

\usepackage[
pdftex,                    % Paramétrage de la navigation
bookmarks = true,          % Signets
bookmarksnumbered = true,  % Signets numérotés
pdfpagemode = None,        % None, UseThumbs, UseOutlines, Fullscreen
pdfstartview = FitH,       % FitH, FitV, FitR, FitB, FitBH, FitBV, Fit
pdfpagelayout = OneColumn, % SinglePage, OneColumn, TwoColumnLeft, TwoColumnRight
colorlinks = true,         % Liens en couleur
urlcolor = blue,           % Couleur des liens externes
pdfborder = {0 0 0}        % Style de bordure : ici, rien
]{hyperref}

\hypersetup{                                     
pdfauthor = {Thomas PABST},% Auteurs
pdftitle = {Le messager de l'E2L qui annonce toute l'année le printemps numérique}, % Titre du document
pdfsubject = {Actualités du libre},    % Sujet
pdfkeywords = {latex,pdf,e2l,E2L, e-rondelle}, % Mots-clefs
pdfcreator = {vim},                            % Logiciel qui a crée le document
pdfproducer = {Patrick Foubet},                % Société avec produit le logiciel
plainpages = true}
\usepackage{pdfpages}


%print Mode or pdfMode

%---EN-TETE-ET-PIED-DE-PAGE----------------------------------------------------
\pagestyle{fancy}

\renewcommand{\footrulewidth}{0.5pt}
\fancyfoot[C]{\textbf{\textit{Page \thepage}}}
\fancyfoot[L]{\textbf{\textit{L'e-Rondelle, Bulletin de l'E2L}}}
\fancyfoot[R]{\textbf{\textit{Numéro 1 du 05 septembre 2015}}}

\begin{document}
\begin{spacing}{1.2}

\begin{center}
\includegraphics[scale=1.0]{pictures/logo-eRondelle.png}\\
\end{center}

\vspace{-3.5cm}

\begin{center}
    \textbf{\textit{\color{blue}
    Le messager de l'E2L qui annonce toute l'année le printemps numérique}}
    \begin{spacing}{1.0}
\vspace{0.2cm}
    \footnotesize{\textbf{
        Bulletin apériodique de l'Ecole du Logiciel Libre\\
        79 avenue Danielle Casanova - 94200 Ivry-sur-Seine\\
        www.e2l.org - Directeur de la Publiction~: Patrick Foubet}}
    \end{spacing}
\end{center}

\vspace{-0.8cm}
\begin{changemargin}{0.8cm}{0.0cm}
    \Huge{\color{RoyalBlue}\textbf{Editorial} \color{black}\large{\textbf{Par Albert Camus}}}
\vspace*{0.1cm}

\normalsize{}
\textbf{\textit{Le monde est ce qu'il est, c'est-à-dire peu de chose. C'est ce que
    chacun sait depuis hier grâce au formidable concert que la radio, les
    journaux et les agences d'information viennent de déclencher au sujet de la
    bombe atomique.}}
\end{changemargin}

\vspace*{-0.1cm}

\fcolorbox{RoyalBlue}{Lavender}{
\begin{minipage}[c][19cm][t]{0.26\linewidth}
    \vspace*{0.4cm}
    \begin{center}
        \Huge{\color{RoyalBlue}\textbf{Sommaire}}
    \end{center}
    
\large{\textbf{\color{darkgray}Actualités}}

\normalsize{}\color{black}
\begin{changemargin}{0.4cm}{0.2cm}
\textbf{p.~06} Shinken v2, vers plus de flexibilité
\end{changemargin}
\vspace*{0.2cm}

\large{\textbf{\color{darkgray}Humeur}}

\normalsize{}\color{black}
\begin{changemargin}{0.4cm}{0.2cm}
\textbf{p.~12} Internet \& élections : tout n’est pas permis
\end{changemargin}
\vspace*{0.2cm}

\large{\textbf{\color{darkgray}Repères}}

\normalsize{}\color{black}
\begin{changemargin}{0.4cm}{0.2cm}
\textbf{p.~16} Une histoire de l'informatique \#6 -La naissance des nains surpuissants
\end{changemargin}
\vspace*{0.2cm}

\large{\textbf{\color{darkgray}Algo~/~IA}}

\normalsize{}\color{black}
\begin{changemargin}{0.4cm}{0.2cm}
\textbf{p.~24} Les expressions régulières ; la théorie et les normes
\end{changemargin}
\vspace*{0.2cm}

\large{\textbf{\color{darkgray}Sysadmin~/Netadmin}}

\normalsize{}\color{black}
\begin{changemargin}{0.4cm}{0.2cm}
\textbf{p.~30} Mise en oeuvre de Ceph
\textbf{p.~40} Supervision distribuée avec Monit (et Puppet)
    \end{changemargin}
    \vspace*{0.2cm}

    \large{\textbf{\color{darkgray}Android}}

    \normalsize{}\color{black}
    \begin{changemargin}{0.4cm}{0.2cm}
    \textbf{p.~66} Android TV et TV Input Framework
    \end{changemargin}
    \vspace*{0.2cm}

    \vspace*{0.4cm}
    \begin{center}
        \includegraphics[scale=1.0]{pictures/logo-eRondelle2.png}\\
    \end{center}

\end{minipage}
}\hfill
\begin{minipage}[c][20cm][t]{0.64\linewidth}
    \begin{multicols}{2}

    \lettrine{\color{RoyalBlue}O\color{black}}n nous apprend, en effet, au
    milieu d'une foule de commentaires enthousiastes que n'importe quelle ville
    d'importance moyenne peut être totalement rasée par une bombe de la grosseur
    d'un ballon de football. Des journaux américains, anglais et français se
    répandent en dissertations élégantes sur l'avenir, le passé, les inventeurs,
    le coût, la vocation pacifique et les effets guerriers, les conséquences
    politiques et même le caractère indépendant de la bombe atomique.  Nous nous
    résumerons en une phrase : la civilisation mécanique vient de parvenir à son
    dernier degré de sauvagerie. Il va falloir choisir, dans un avenir plus ou
    moins proche, entre le suicide collectif ou l'utilisation intelligente des
    conquêtes scientifiques.\\


    \lettrine{\color{RoyalBlue}E\color{black}}n attendant, il est permis de penser
    qu'il y a quelque indécence à célébrer ainsi une découverte, qui se met d'abord
    au service de la plus formidable rage
    de destruction dont l'homme ait fait preuve depuis des siècles. Que dans un
    monde livré à tous les déchirements de la violence, incapable d'aucun contrôle,
    indifférent à la justice et au simple bonheur des hommes, la science se consacre
    au meurtre organisé, personne sans doute, à moins d'idéalisme impénitent, ne
    songera à s'en étonner.  Les découvertes doivent être enregistrées, commentées
    selon ce qu'elles sont, annoncées au monde pour que l'homme ait une juste idée
    de son destin. Mais entourer ces terribles révélations d'une littérature
    pittoresque ou humoristique, c'est ce qui n'est pas supportable. Déjà, on ne
    respirait pas facilement dans un monde torturé. Voici qu'une angoisse nouvelle
    nous est proposée, qui a toutes les chances d'être définitive. On offre sans
    doute à l'humanité sa dernière chance. Et ce peut-être après tout le prétexte
    d'une édition spéciale. Mais ce devrait être plus sûrement le sujet de quelques
    réflexions et de beaucoup de silence.\\


    \lettrine{\color{RoyalBlue}A\color{black}}u reste, il est d'autres raisons
    d'accueillir avec réserve le roman d'anticipation que les journaux nous
    proposent. Quand on voit le rédacteur
    diplomatique de l'Agence Reuter* annoncer que cette invention rend caducs les
    traités ou périmées les décisions mêmes de Potsdam*, remarquer qu'il est
    indifférent que les Russes soient à Koenigsberg ou la Turquie aux Dardanelles,
    on ne peut se défendre de supposer à ce beau concert des intentions assez
    étrangères au désintéressement scientifique. Qu'on nous entende bien. Si les
    Japonais capitulent après la destruction d'Hiroshima et par l'effet de
    l'intimidation, nous nous en réjouirons. Mais nous nous refusons à tirer d'une
    aussi grave nouvelle autre chose que la décision de plaider plus énergiquement
    encore en faveur d'une véritable société internationale, où les grandes
    puissances n'auront pas de droits supérieurs aux petites et aux moyennes
    nations, où la guerre, fléau devenu définitif par le seul effet de
    l'intelligence humaine, ne dépendra plus des appétits ou des doctrines de tel ou
    tel État.
\end{multicols}

\pagebreak



\end{minipage}
\end{spacing}


\begin{spacing}{1.2}
\vspace*{0.1cm}
\begin{multicols}{2}
\lettrine{\color{RoyalBlue}A\color{black}} percevons encore mieux que la paix
est le seul combat qui vaille d'être mené.  Ce n'est plus une prière, mais un
ordre qui doit monter des peuples vers les gouvernements, l'ordre de choisir
définitivement entre l'enfer et la raison.  \end{multicols}

\end{spacing}
\end{document}
